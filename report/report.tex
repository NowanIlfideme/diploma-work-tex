% !TeX spellcheck = en_US

\documentclass[a4paper,14pt]{extreport}


\usepackage{docdef}


\addbibresource{../bibliography/sources.bib}

\begin{document}
	\maketitle
	
	\iffalse
	{
		\thispagestyle{empty}
	
		TODO: Задание -- приложение 2
		
		Задание на дипломную работу (дипломный проект), оформленное
		на типовом бланке, подписанное обучающимся, руководителем и утвержденное
		заведующим кафедрой (Приложение 2). Номер страницы на «задание на
		дипломную работу (дипломный проект)» не ставится и оно не включается в
		общую нумерацию страниц;
		
		Т.е. отдельно? Хмм.
		
		\clearpage
	}
	\fi	
	
	% Название содержания меняем
	\renewcommand{\contentsname}{Содержание}
	\tableofcontents
	
	\clearpage
	
	\iffalse
	\bsuabstract
	{
		This paper examines methods for extraction of economic cycles, identification of their turning points, and prediction of future values. This paper also compares the results of these methods when used to identify cycles in the GDP of the Republic of Belarus. Methods reviewed: double Hodrick-Prescott filtering, Hamilton's proposed method,  Markov-switching autoregressive models (MS-ARX), and SARIMAX models. The work also includes a short description of the development of a Python-language library for time series analysis and modelling, which includes implementations of the above models.
		
	}{
		В данной работе рассматриваются методы выделения экономических циклов, определения их поворотных точек, и предсказания будущих значений. Также сравниваются результаты применения этих методов при идентификации циклов в ВВП Республики Беларусь. Рассмотренные методы: двойное использование фильтра Ходрика - Прескотта, метод Хамильтона, авторегрессионные модели с Марковским переключением состояний (MS-ARX), модели SARIMAX. В работе также приводится краткое описание разрабатываемой библиотеки программ на языке Python, предназдаченной для анализа и моделирования временных рядов. Библиотека включает разделы, посвященные вышеописанным моделям.
	}
	\fi
	
	%\iffalse
	
	\bsureferatru{\cntpages страниц, \cntimages рисунков, \cnttables таблиц, \cntsources источников}
	{
		В работе представляются результаты, полученные автором при решении задач, связанных с построением и применением моделей с марковскими переключениями состояний из семейства MS-VARX и сезонной ARIMA-модели (SARIMAX), использующих индекс экономических настроений и индексы доверия белорусской экономики на основе опросных данных системы мониторинга Национального банка Республики Беларусь.
		
		Исследования в данном направлении проводились в Белорусском государственном университете в 2016-2017 гг. в рамках НИР "Разработка системы опережающих экономических индикаторов и экономических диффузных индексов для основных видов экономической деятельности и экономики Республики Беларусь в целом с использованием экономико-математических, эконометрических методов и моделей на основе данных системы мониторинга предприятий Национального банка Республики Беларусь".
		
		Модельный и программный инструментарий для построения указанных  индексов и их применения в предиктивных эконометрических моделях для реального ВВП, а также в моделях с марковскими переключениями состояний для анализа бизнес-цикла белорусской экономики представлены в заключительном отчете о НИР \cite{esiMaking}.
	}
	\bsureferaten{\cntpages pages, \cntimages images, \cnttables tables, \cntsources sources}
	{
		TODO English!
		В работе представляются результаты, полученные автором при решении задач, связанных с построением и применением моделей с марковскими переключениями состояний из семейства MS-VARX и сезонной ARIMA-модели (SARIMAX), использующих индекс экономических настроений и индексы доверия белорусской экономики на основе опросных данных системы мониторинга Национального банка Республики Беларусь.
		
		Исследования в данном направлении проводились в Белорусском государственном университете в 2016-2017 гг. в рамках НИР "Разработка системы опережающих экономических индикаторов и экономических диффузных индексов для основных видов экономической деятельности и экономики Республики Беларусь в целом с использованием экономико-математических, эконометрических методов и моделей на основе данных системы мониторинга предприятий Национального банка Республики Беларусь".
		
		Модельный и программный инструментарий для построения указанных  индексов и их применения в предиктивных эконометрических моделях для реального ВВП, а также в моделях с марковскими переключениями состояний для анализа бизнес-цикла белорусской экономики представлены в заключительном отчете о НИР \cite{esiMaking}.
	}
	%\fi
	
	\clearpage
	
	\subfile{sections/contents}
	
	\clearpage
	
	\bsuconclusion{
		Анализ экономических циклов и оценка моментов смены фаз циклов, называемых поворотными точками, является одной из актуальных проблем макроэкономического анализа и прогнозирования. Для проведения такого анализа используется некоторый базовый экономический индикатор, характеризующий состояние экономики, относительно которого рассчитываются показатели ее роста и спада. В качестве такого индикатора обычно используется реальный ВВП. Для прогнозирования моментов смены фаз реального ВВП традиционно применяют два подхода. 
		
		Первый подход основан на построении и применении опережающих экономических индикаторов. Индикатор считается опережающим, если поворотные точки его цикла (точки переключения фаз цикла «спад» и «подъем») опережают поворотные точки  цикла базового экономического индикатора. В рамках данного подхода ключевыми являются две задачи: задача сезонной корректировки временных рядов базового и опережающего индикаторов и задача выделения циклической составляющей из сезонно-скорректированных временных рядов. На основе сравнения циклов для  базового и опережающего индикаторов осуществляется оценка и прогнозирование поворотных точек базового индикатора по поворотным точкам опережающего индикатора.
		
		Второй подход основан на применении эконометрических моделей с марковскими переключениями состояний. Обычно используются модели одномерной и векторной авторегрессии с марковскими переключениями состояний (MS-VAR). Для оценки поворотных точек в рамках данного подхода используются алгоритмы совместного оценивания номеров классов состояний экономики («спад»/«подъем») и параметров моделей.
		
		В качестве базового индикатора в работе используется месячный реальный ВВП Республики Беларусь в ценах 2014 года, а в качестве опережающего – индекс экономических настроений (ИЭН) Республики Беларусь, построенный на основе опросных данных системы мониторинга предприятий Национального банка Республики Беларусь. Методика его построения, а также разработанные модельные и программные средства представлены в  заключительном отчете о НИР, в рамках которой проводилось и данное исследование. Там же установлено, что поворотные точки построенного ИЭН опережают поворотные точки реального  ВВП на 4–5 месяцев.
		
		В данной работе решалось три основные задачи:
		\begin{enumerate}
			
			\item   сравнительный анализ двух методов выделения циклической составляющей временного ряда, включая: двойное применение фильтра Ходрика – Прескотта (традиционный подход) и нового метода, предложенного Дж. Хамильтоном в 2017 г.  и пока не имеющего большой практики применения;
			\item   построение моделей с марковскими переключениями состояний из семейства MS-VARX и сезонной модели ARIMAX, включающих в качестве экзогенной переменной опережающий индекс и ее применение для анализа экономического цикла  и оценки поворотных точек;
			\item   разработка библиотеки программ на языке Python для анализа временных рядов, включающая использованные в работе эконометрические модели и методы анализа.
		\end{enumerate}
		
		Все решаемые задачи являются новыми и имеют практическую значимость. Анализ циклов и их поворотных точек, полученных при решении двух указанных задач, а также их сравнение с экспертными оценками, полученными в рамках указанной НИР, свидетельствуют о том, что поворотные точки циклов либо совпадают, либо отличаются на 1-2 месяца для различных способов оценки. Главное несоответствие  возникает в периоде 2011-2013 гг., который характеризуется высокой неопределенностью экономической конъюнктуры.
		
		Таким образом, к числу основных результатов работы можно отнести следующие:
		
		\begin{itemize}
			\item исслледованы возможности применения методов Ходрика - Прескотта и Хамильтона для выделения циклов в экономических временных рядах;
			\item на основе этих методов оценены поворотные точки бизнес-цикла белорусской экономики и проведено их сравнене с экспертными оценками;
			\item построены модели с переключением состояний MS-ARX для временного ряда реального ВВП Республики Беларусь, и эти модели использованы для анализа бизнес-цикла;
			\item исследованы предиктивные возможности моделей MS-ARX и SARIMAX для реального ВВП;
			\item спроектирована и разработана библиотека программ \textbf{ts4ds} на языке Python для анализа временных рядов ;
			\item исходный код и данные  первых трех глав работы размещены в открытом доступе на сайте {https://github.com/NowanIlfideme/PyEconModelling}, однако исходный код библиотеки на данный момент еще не опубликован.
		\end{itemize}
		
		\iffalse
		Сделаны выводы:
		
		\begin{itemize}
			\item Метод выделения циклов Хамильтона применима как альтернатива фильтру Ходрика-Прескотта для ВВП и ИЭН Республики Беларусь;
			\item Перед применением этого метода, желательно убирать сезонность, так как иначе остаются сезонные эффекты;
			\item Ряды, полученные по этому методу, имеют свойства схожие с другими методами детрендирования, но без ложных автокорреляций, и по ним возможно построить модель MS-ARX со схожими поворотными точками;
			\item Прогнозная точность моделей MS-ARX уступает другим моделям, следовательно их стоит использовать преимущественно при ретроспективном анализе.
			\item Отделение моделей и алгоритмов оценивания упрощает работу с моделями;
			\item Выделение отдельной библиотеки и стандартизации интерфейса способствует упрощению программного кода и ускоряет разработку и тестирование новых моделей.
		\end{itemize}
		\fi
	}
	
	% fake chapter чтобы список лит был на одной странице
	\chapter*{}
	\addcontentsline{toc}{chapter}{Список источников}
	\markboth{Список источников}{Список источников}
	\printbibliography[title=Список источников]
	
	\appendix
\end{document}