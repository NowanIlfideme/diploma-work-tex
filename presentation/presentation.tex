
\documentclass{beamer}

% support unicode
\input glyphtounicode
\pdfgentounicode=1
\pdfmapfile{=cm-super-t2a.map}

% support specifically Russian
\usepackage{lmodern}
\usepackage[utf8]{inputenc}
\usepackage[T1,T2A]{fontenc}
\usepackage[english,russian]{babel}
\selectlanguage{\russian}


% STYLE
\usetheme{Malmoe}
\usecolortheme{default}
\useoutertheme{tree}

% Uncomment to remove nav
%\setbeamertemplate{navigation symbols}{}

% Code for section beginning

%\AtBeginSection[]
%{
%	\begin{frame}
%		\tableofcontents[currentsection]
%	\end{frame}
%}



% Set picture materials
%\usepackage{graphicx} % note - beamer already imports
\graphicspath{ {../materials/img/} }



% META-INFO
\title{РАЗРАБОТКА И ПРИМЕНЕНИЕ МОДЕЛЕЙ MS-VARX ДЛЯ АНАЛИЗА ЭКОНОМИЧЕСКИХ ЦИКЛОВ}
\author{Макаревич А.С.}
\institute[БГУ]{Белорусский Государственный Университет}
\date[Минск, 2018]{Минск, 2018}
\subject{text}


% 
\begin{document}


	% Title page
	\begin{frame}
		\titlepage
	\end{frame}

	\begin{frame}
		\tableofcontents[]
	\end{frame}

	\section{Экономические циклы и подходы к их анализу. Цель исследования}
		\subsection{Фазы экономического цикла и поворотные точки: концепции НБЭИ и ОСЭРs}
		\begin{frame}
			\frametitle{Фазы экономического цикла и поворотные точки: концепции НБЭИ и ОСЭР}
			
			TODO: Описать фазы.
			
			Экономический цикл (по НБЭИ США)
			\begin{itemize}
				\item фазы: <<рост>> и <<спад>>
				\item точки: <<пик>> и <<дно>>
			\end{itemize}
	
		\end{frame}

		\subsection{Используемые данные, цель исследования и решаемые задачи}
		\begin{frame}
			\frametitle{Используемые данные, цель исследования и решаемые задачи}
			
			Используемые данные
			
			В качестве базового экономического индикатора используется реальный месячный ВВП (в ценах 2014 г.).
			
			В качестве опережающего экономического индикатора используется индекс экономических настроений (ИЭН) белорусской экономики, построенный по опросным данным белорусских предприятий, получаемым в рамках  системы мониторинга Национального банка Республики Беларусь.
			
			Период наблюдения: май 2005 г. – январь 2017 г.
			
			Цель исследования исследовать возможности использования ИЭН для анализа цикла бе-лорусской экономики на основе  эконометрических методов и моделей.
			
			Основные задачи исследования:
			
			\begin{itemize}
				\item Провести сравнительный анализ двух методов выделения циклической составляющей временного ряда:
				\begin{enumerate}
					первый метод (традиционный) основан на двойном применение фильтра Ходрика – Пре-скотта;
					второй  метод предложен Дж. Хамильтон предложен в 2017 г. и ранее не применялся бело-русскими исследователями.
				\end{enumerate}
				\item Исследовать возможность использования ИЭН в модели с марковскими пере-ключениями  состояний для анализа экономического цикла белорусской экономики.
				\item Разработать специализированную библиотеку программ на языке Python для анализа временных рядов
			
		\end{frame}

		\subsection{Эконометрические методы и модели для анализа экономических циклов и поворотных точек}
		\begin{frame}
			\frametitle{Сравнительный анализ циклов базового и опережающего экономических индикаторов}
			
			TODO: Описание общей задачи.
			
			TODO: Описать результаты (4-5 месяца)
			
		\end{frame}
		
		
		\begin{frame}
			\frametitle{Использование моделей с марковскими переключениями со-стояний MS-VARX.}
						
			
			\begin{itemize}
				\item Сезонная корректировка (X13-ARIMA-SEATS): тренд + цикл, сезонность, шум
				
				\item Фильтр Ходрика--Прескотта: 
					\begin{itemize}
						\item низкочастотный: тренд, цикл, шум
						\item высокочастотный: тренд (или цикл), шум
					\end{itemize}
				
				\item Метод Хамильтона: тренд + шум, цикл
			\end{itemize}
			
		\end{frame}
	
	
					
	\section{Анализ цикла белорусской экономики на основе методов Ходрика -- Прескотта и Хамильтона}
		\subsection{Описание методов и результаты их применения }
		\begin{frame}
			\frametitle{Описание методов Ходрика -- Прескотта и Хамильтона}
			
			
			Режим -- состояние системы (например, фазы <<роста>> и <<падения>>).
			
			Каждому режиму соотносится своя построенная модель.
			
			Переключение режимов происходит по какому--то правилу. Из относительно простых -- независимое переключение ($IS$) и цепь Маркова ($MS$).
			
			В качестве базовой модели -- $ARX$, в кач. экзогенной -- ИЭН.
			
			$ \Rightarrow $ модель $ MS-ARX $
			
		\end{frame}
	
		\begin{frame}
			\frametitle{Результаты применения методов}
			текст, графики
		\end{frame}
	
		\subsection{Сравнительный анализ поворотных точек реального ВВП и индекса экономических настроения (ИЭН) белорусской экономики}
		
	\section{Модели семейства MS-VARX, использующие ИЭН, и их применение для анализа цикла белорусской экономики}
		\subsection{Описание модели MS-VARX}
		\begin{frame}
			\frametitle{Описание модели MS-VARX}
			
			
		\end{frame}


		\subsection{Частные случаи модели: MS-AR  и MS-ARX }
		\begin{frame}
			\frametitle{Частные случаи модели: MS-AR  и MS-ARX }
			
			
		\end{frame}


		
		\subsection{Модель MS-ARX, использующая ИЭН в качестве экзогенной пере-менной  и результаты ее применения }
		\begin{frame}
			\frametitle{Модель MS-ARX, использующая ИЭН в качестве экзогенной пере-менной  и результаты ее применения }
			
		\end{frame}
		
		
		
		
	\section{Разработанная библиотека программ на языке Python}
		\begin{frame}
			\frametitle{Разработанная библиотека программ на языке Python}
			
			
		\end{frame}
		
		
	
	\section*{Заключение}
	\begin{frame}
		\frametitle{Заключение}
	\end{frame}


	\section*{Список использованных источников }
		\begin{frame}
		\frametitle{Список использованных источников }
	\end{frame}
	
\end{document}