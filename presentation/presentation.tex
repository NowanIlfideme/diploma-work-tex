
\documentclass{beamer}

% support unicode
\input glyphtounicode
\pdfgentounicode=1
\pdfmapfile{=cm-super-t2a.map}

% support specifically Russian
\usepackage{lmodern}
\usepackage[utf8]{inputenc}
\usepackage[T1,T2A]{fontenc}
\usepackage[english,russian]{babel}
\selectlanguage{\russian}


% STYLE
\usetheme{Malmoe}
\usecolortheme{default}
\useoutertheme{tree}

% Uncomment to remove nav
%\setbeamertemplate{navigation symbols}{}

% Code for section beginning
\AtBeginSection[]
{
	\begin{frame}
		\frametitle{План доклада}
		\tableofcontents[currentsection]
	\end{frame}
}



% META-INFO
\title{РАЗРАБОТКА И ПРИМЕНЕНИЕ МОДЕЛЕЙ MSVARX ДЛЯ АНАЛИЗА ЭКОНОМИЧЕСКИХ ЦИКЛОВ}
\author{Макаревич А.С.}
\institute[БГУ]{Белорусский Государственный Университет}
\date[Минск, 2018]{Минск, 2018}
\subject{text}


% 
\begin{document}


	% Title page
	\begin{frame}
		\titlepage
	\end{frame}

	\section{Описание задач}
		\subsection{Анализ экономических циклов}
		\begin{frame}
			\frametitle{Задача анализа экономических циклов}
			
			\begin{itemize}
				\item Базовый ряд -- Реальный ВВП РБ (цены 2014 г.)
				\item Экономический цикл (по НБЭИ США)
					\begin{itemize}
						\item объект: тренд ВВП
						\item фазы: <<рост>> и <<спад>>
						\item точки: <<пик>> и <<дно>>
						\end{itemize}
				\pause
				\item Опережающий индикатор -- Индекс Экономических Настроений (ИЭН)
					\begin{itemize}
						\item по данным сист. мониторинга предприятий НБ РБ
						\item циклы опережают ВВП на 4-5 месяца
					\end{itemize}
			\end{itemize}
			
		\end{frame}

		\subsection{Декомпозиция временных рядов}
		\begin{frame}
			\frametitle{Декомпозиция временных рядов}
			
			Хамильтона
			
		\end{frame}
	
		\subsection{Модели с переключением состояний}
		\begin{frame}
			\frametitle{Модели с переключением состояний}
			
			По MS-ARX
			
		\end{frame}
	
	\section{Экспериментальные исследования}
		\subsection{Корректировка по Хамильтону}
		\begin{frame}
			\frametitle{Корректировка по Хамильтону}
			
			Картинки по Хам
			
		\end{frame}

		\subsection{Моделирование с MS-ARX}
		\begin{frame}
			\frametitle{Моделирование с MS-ARX}
			
			Картинки режимов
			
		\end{frame}
	
	\section{Разработка библиотеки TS4DS}
		\begin{frame}
			\frametitle{Библиотека TS4DS}
			
			\begin{enumerate}
				\item content...
				\item c2
			\end{enumerate}
			
		\end{frame}

	
\end{document}