\documentclass[../main.tex]{subfiles}
\begin{document}
	
	\bsuconclusion{
		Анализ экономических циклов и оценка моментов смены фаз циклов, называемых поворотными точками, является одной из актуальных проблем макроэкономического анализа и прогнозирования. Для проведения такого анализа используется некоторый базовый экономический индикатор, характеризующий состояние экономики, относительно которого рассчитываются показатели ее роста и спада. В качестве такого индикатора обычно используется реальный ВВП. Для прогнозирования моментов смены фаз реального ВВП традиционно применяют два подхода. 
		
		Первый подход основан на построении и применении опережающих экономических индикаторов. Индикатор считается опережающим, если поворотные точки его цикла (точки переключения фаз цикла «спад» и «подъем») опережают поворотные точки  цикла базового экономического индикатора. В рамках данного подхода ключевыми являются две задачи: задача сезонной корректировки временных рядов базового и опережающего индикаторов и задача выделения циклической составляющей из сезонно-скорректированных временных рядов. На основе сравнения циклов для  базового и опережающего индикаторов осуществляется оценка и прогнозирование поворотных точек базового индикатора по поворотным точкам опережающего индикатора.
		
		Второй подход основан на применении эконометрических моделей с марковскими переключениями состояний. Обычно используются модели одномерной и векторной авторегрессии с марковскими переключениями состояний (MS-VAR). Для оценки поворотных точек в рамках данного подхода используются алгоритмы совместного оценивания номеров классов состояний экономики («спад»/«подъем») и параметров моделей.
		
		В качестве базового индикатора в работе используется месячный реальный ВВП Республики Беларусь в ценах 2014 года, а в качестве опережающего – индекс экономических настроений (ИЭН) Республики Беларусь, построенный на основе опросных данных системы мониторинга предприятий Национального банка Республики Беларусь. Методика его построения, а также разработанные модельные и программные средства представлены в  заключительном отчете о НИР, в рамках которой проводилось и данное исследование. Там же установлено, что поворотные точки построенного ИЭН опережают поворотные точки реального  ВВП на 4–5 месяцев.
		
		В данной работе решалось три основные задачи:
		\begin{enumerate}
			
		\item   сравнительный анализ двух методов выделения циклической составляющей временного ряда, включая: двойное применение фильтра Ходрика – Прескотта (традиционный подход) и нового метода, предложенного Дж. Хамильтоном в 2017 г.  и пока не имеющего большой практики применения;
		\item   построение моделей с марковскими переключениями состояний из семейства MS-VARX и сезонной модели ARIMAX, включающих в качестве экзогенной переменной опережающий индекс и ее применение для анализа экономического цикла  и оценки поворотных точек;
		\item   разработка библиотеки программ на языке Python для анализа временных рядов, включающая использованные в работе эконометрические модели и методы анализа.
		\end{enumerate}

		Все решаемые задачи являются новыми и имеют практическую значимость. Анализ циклов и их поворотных точек, полученных при решении двух указанных задач, а также их сравнение с экспертными оценками, полученными в рамках указанной НИР, свидетельствуют о том, что поворотные точки циклов либо совпадают, либо отличаются на 1-2 месяца для различных способов оценки. Главное несоответствие  возникает в периоде 2011-2013 гг., который характеризуется высокой неопределенностью экономической конъюнктуры.
		
		Таким образом, к числу основных результатов работы можно отнести следующие:
		
		\begin{itemize}
			\item исслледованы возможности применения методов Ходрика - Прескотта и Хамильтона для выделения циклов в экономических временных рядах;
			\item на основе этих методов оценены поворотные точки бизнес-цикла белорусской экономики и проведено их сравнене с экспертными оценками;
			\item построены модели с переключением состояний MS-ARX для временного ряда реального ВВП Республики Беларусь, и эти модели использованы для анализа бизнес-цикла;
			\item исследованы предиктивные возможности моделей MS-ARX и SARIMAX для реального ВВП;
			\item спроектирована и разработана библиотека программ \textbf{ts4ds} на языке Python для анализа временных рядов ;
			\item исходный код и данные  первых трех глав работы размещены в открытом доступе на сайте {https://github.com/NowanIlfideme/PyEconModelling}, однако исходный код библиотеки на данный момент еще не опубликован.
		\end{itemize}
	
		\iffalse
		Сделаны выводы:
		
		\begin{itemize}
			\item Метод выделения циклов Хамильтона применима как альтернатива фильтру Ходрика-Прескотта для ВВП и ИЭН Республики Беларусь;
			\item Перед применением этого метода, желательно убирать сезонность, так как иначе остаются сезонные эффекты;
			\item Ряды, полученные по этому методу, имеют свойства схожие с другими методами детрендирования, но без ложных автокорреляций, и по ним возможно построить модель MS-ARX со схожими поворотными точками;
			\item Прогнозная точность моделей MS-ARX уступает другим моделям, следовательно их стоит использовать преимущественно при ретроспективном анализе.
			\item Отделение моделей и алгоритмов оценивания упрощает работу с моделями;
			\item Выделение отдельной библиотеки и стандартизации интерфейса способствует упрощению программного кода и ускоряет разработку и тестирование новых моделей.
		\end{itemize}
		\fi
	}
\end{document}