\documentclass[../main.tex]{subfiles}
\begin{document}
	
	\bsuconclusion{
		В работе получены следующие основные результаты:
		
		\begin{itemize}
			\item ислледованы возможности применения методов Ходрика - Прескотта и Хамильтона для выделения циклов в экономических временных рядах;
			\item на основе этих методов оценены поворотные точки бизнес-цикла белорусской экономики и проведено их сравнене с экспертными оценками;
			\item построены модели с переключением состояний MS-ARX для временного ряда реального ВВП Республики Беларусь, и эти модели использованы для анализа бизнес-цикла;
			\item исследованы предиктивные возможности моделей MS-ARX и SARIMAX для реального ВВП;
			\item спроектирована и разработана библиотека программ \textbf{ts4ds} на языке Python для анализа временных рядов ;
			\item исходный код и данные  первых трех глав работы размещены в открытом доступе на сайте {https://github.com/NowanIlfideme/PyEconModelling}, однако исходный код библиотеки на данный момент еще не опубликован.
		\end{itemize}
	
		\iffalse
		Сделаны выводы:
		
		\begin{itemize}
			\item Метод выделения циклов Хамильтона применима как альтернатива фильтру Ходрика-Прескотта для ВВП и ИЭН Республики Беларусь;
			\item Перед применением этого метода, желательно убирать сезонность, так как иначе остаются сезонные эффекты;
			\item Ряды, полученные по этому методу, имеют свойства схожие с другими методами детрендирования, но без ложных автокорреляций, и по ним возможно построить модель MS-ARX со схожими поворотными точками;
			\item Прогнозная точность моделей MS-ARX уступает другим моделям, следовательно их стоит использовать преимущественно при ретроспективном анализе.
			\item Отделение моделей и алгоритмов оценивания упрощает работу с моделями;
			\item Выделение отдельной библиотеки и стандартизации интерфейса способствует упрощению программного кода и ускоряет разработку и тестирование новых моделей.
		\end{itemize}
		\fi
	}
\end{document}