\documentclass[../main.tex]{subfiles}
\begin{document}
	
	\bsuabstract
	{
		This paper examines methods for extraction of economic cycles, identification of their turning points, and prediction of future values. This paper also compares the results of these methods when used to identify cycles in the GDP of the Republic of Belarus. Methods reviewed: double Hodrick-Prescott filtering, Hamilton's proposed method,  Markov-switching autoregressive models (MS-ARX), and SARIMAX models. The work also includes a short description of the development of a Python-language library for time series analysis and modelling, which includes implementations of the above models.
		
	}{
		В данной работе рассматриваются методы выделения экономических циклов, определения их поворотных точек, и предсказания будущих значений. Также сравниваются результаты применения этих методов при идентификации циклов в ВВП Республики Беларусь. Рассмотренные методы: двойное использование фильтра Ходрика - Прескотта, метод Хамильтона, авторегрессионные модели с Марковским переключением состояний (MS-ARX), модели SARIMAX. В работе также приводится краткое описание разрабатываемой библиотеки программ на языке Python, предназдаченной для анализа и моделирования временных рядов. Библиотека включает разделы, посвещенные вышеописанным моделям.
	}
	
	\bsureferat{39 страниц, 16 рисунков, 4 таблиц, 23 источников}
	{
		В работе представляются результаты, полученные автором при решении задач, связанных с построением и применением моделей с марковскими переключениями состояний из семейства MS-VARX и сезонной ARIMA-модели (SARIMAX), использующих индекс экономических настроений и индексы доверия белорусской экономики на основе опросных данных системы мониторинга Национального банка Республики Беларусь.
		
		Исследования в данном направлении проводились в Белорусском государственном университете в 2016-2017 гг. в рамках НИР "Разработка системы опережающих экономических индикаторов и экономических диффузных индексов для основных видов экономической деятельности и экономики Республики Беларусь в целом с использованием экономико-математических, эконометрических методов и моделей на основе данных системы мониторинга предприятий Национального банка Республики Беларусь".
		
		Модельный и программный инструментарий для построения указанных  индексов и их применения в предиктивных эконометрических моделях для реального ВВП, а также в моделях с марковскими переключениями состояний для анализа бизнес-цикла белорусской экономики представлены в заключительном отчете о НИР \cite{esiMaking}.
	}	
\end{document}